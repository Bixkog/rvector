% Opcje klasy 'iithesis' opisane sa w komentarzach w pliku klasy. Za ich pomoca
% ustawia sie przede wszystkim jezyk i rodzaj (lic/inz/mgr) pracy, oraz czy na
% drugiej stronie pracy ma byc skladany wzor oswiadczenia o autorskim wykonaniu.
\RequirePackage{color}
\documentclass[inz, english, shortabstract]{iithesis}

\usepackage[utf8]{inputenc}
\usepackage{listings, multicol}
\usepackage{enumitem}
\usepackage{amsmath}
\usepackage{graphicx}
\usepackage{textcomp}

%%%%% DANE DO STRONY TYTUŁOWEJ
% Niezaleznie od jezyka pracy wybranego w opcjach klasy, tytul i streszczenie
% pracy nalezy podac zarowno w jezyku polskim, jak i angielskim.
% Pamietaj o madrym (zgodnym z logicznym rozbiorem zdania oraz estetyka) recznym
% zlamaniu wierszy w temacie pracy, zwlaszcza tego w jezyku pracy. Uzyj do tego
% polecenia \fmlinebreak.
\polishtitle    {Szybki vector z mremap}
\englishtitle   {A fast vector with mremap}
\polishabstract {W języku C++ jedną z najczęściej używanych struktur danych jest wektor, który jest rozwinięciem tablic dynamicznych stosowanych w C. Oprócz automatycznego zarządzania pamięcią, pozwala on na dynamiczne zwiększanie rozmiaru trzymanej tablicy, realokując trzymany blok pamięci kiedy jest to wymagane. Implementacja tej struktury znajdująca się w bibliotece standardowej ({\it std::vector}) wykonuje tą realokacje wykorzystując schemat {\it allocate, move, deallocate}. Jednak wykorzystując interfejs wywołania systemowego {\it mremap}\cite{mremap} można sprawdzić czy system operacyjny nie byłby w stanie rozszerzyć danego bloku pamięci w miejscu, co pozwoliłoby pominąć cały proces realokacji. {\it Mremap} pozwala także na realokację obszaru pamięci w czasie stałym poprzez zmianę mapowania adresu wirtualnego na fizyczny. Jeśli typ danych w wektorze jest trywialnie przenoszony, to można wykorzystać tą możliwość do szybkiej realokacji bloku pamięci w wektorze. Tę optymalizację oraz kilka innych, zawiera napisany przeze mnie {\it rvector}.}
\englishabstract{In C++ one of the most commonly used data structure is vector, which is a wrapper of dynamically allocated arrays used in C. Aside from automatic memory management, it allows to dynamically increase size of held array, with reallocation of contained memory block when it is required. STL implementation of this data structure ({\it std::vector}) executes this reallocation utilizing scheme {\it allocate, move, deallocate}. However, by using syscall {\it mremap}\cite{mremap} it is possible to check whether operating system could expand specified memory block in place, which would allow to omit whole reallocation process. {\it Mremap} may be also used to reallocate memory block in constant time by changing virtual address mapping. It can be used to quickly reallocate memory block of trivially movable objects.  My implementation of vector {\it rvector} contains that and few other optimizations.}
% w pracach wielu autorow nazwiska mozna oddzielic poleceniem \and
\author         {Wojciech Oziębły}
% w przypadku kilku promotorow, lub koniecznosci podania ich afiliacji, linie
% w ponizszym poleceniu mozna zlamac poleceniem \fmlinebreak
\advisor        {dr Marek Szykuła}
%\date          {}                     % Data zlozenia pracy
% Dane do oswiadczenia o autorskim wykonaniu
\transcriptnum {281539}                     % Numer indeksu
\advisorgen    {dr. Marka Szykuły} % Nazwisko promotora w dopelniaczu
%%%%%

%%%%% WLASNE DODATKOWE PAKIETY
%
%\usepackage{graphicx,listings,amsmath,amssymb,amsthm,amsfonts,tikz}
%
%%%%% WŁASNE DEFINICJE I POLECENIA
%
%\theoremstyle{definition} \newtheorem{definition}{Definition}[chapter]
%\theoremstyle{remark} \newtheorem{remark}[definition]{Observation}
%\theoremstyle{plain} \newtheorem{theorem}[definition]{Theorem}
%\theoremstyle{plain} \newtheorem{lemma}[definition]{Lemma}
%\renewcommand \qedsymbol {\ensuremath{\square}}
% ...
%%%%%


\begin{document}

\definecolor{mygreen}{rgb}{0,0.6,0}
\definecolor{mygray}{rgb}{0.5,0.5,0.5}
\definecolor{mymauve}{rgb}{0.58,0,0.82}

\lstset{ 
  backgroundcolor=\color{white},   % choose the background color; you must add \usepackage{color} or \usepackage{xcolor}; should come as last argument
  basicstyle=\ttfamily\footnotesize,        % the size of the fonts that are used for the code
  breakatwhitespace=false,         % sets if automatic breaks should only happen at whitespace
  breaklines=true,                 % sets automatic line breaking
  captionpos=b,                    % sets the caption-position to bottom
  commentstyle=\color{mygreen},    % comment style
  deletekeywords={...},            % if you want to delete keywords from the given language
  escapeinside={\%*}{*)},          % if you want to add LaTeX within your code
  extendedchars=true,              % lets you use non-ASCII characters; for 8-bits encodings only, does not work with UTF-8
  frame=none,	                   % adds a frame around the code
  keepspaces=true,                 % keeps spaces in text, useful for keeping indentation of code (possibly needs columns=flexible)
  keywordstyle=\color{blue},       % keyword style
  language=C++,                    % the language of the code
  morekeywords={*,...},            % if you want to add more keywords to the set
  numbers=left,                    % where to put the line-numbers; possible values are (none, left, right)
  numbersep=5pt,                   % how far the line-numbers are from the code
  numberstyle=\tiny\color{mygray}, % the style that is used for the line-numbers
  rulecolor=\color{black},         % if not set, the frame-color may be changed on line-breaks within not-black text (e.g. comments (green here))
  showspaces=false,                % show spaces everywhere adding particular underscores; it overrides 'showstringspaces'
  showstringspaces=false,          % underline spaces within strings only
  showtabs=false,                  % show tabs within strings adding particular underscores
  stepnumber=1,                    % the step between two line-numbers. If it's 1, each line will be numbered
  stringstyle=\color{mymauve},     % string literal style
  tabsize=4,	                   % sets default tabsize to 2 spaces
  title=\lstname                   % show the filename of files included with \lstinputlisting; also try caption instead of title
}

%%%%% POCZĄTEK ZASADNICZEGO TEKSTU PRACY

\chapter{Vector data structure}

Vector is the most commonly used and general data structure in C++. Its purpose is simple, management of continuous dynamic array of objects. Because of its universal use and general applicability, vector is required to be fast, memory efficient and reliable. It is supposed to work great as a container for billions of int objects, as well as for few big abstract objects. As almost all software written in C++ relies heavily on vectors, efficient implementation of this data structure is highly desirable.
\\
Vector stores all objects in a continuous memory block, which allows constant time access to its elements. It allows to dynamically add and erase elements, with storage being automatically expanded when needed. To do it efficiently, capacity of vector is often greater than actual amount of stored elements. Key parameter in any vector implementation is {\it growth factor}, which defines expansion rate of capacity when an additional memory is required. The most common value of growth factor is 2. With such allocation strategy, adding element to the back of vector is amortized constant time. Yet, as the reallocation process is often very costly, it renders vector not suitable for certain tasks, e.g. real-time systems. 

\chapter{Common vector implementations}
\section{std::vector}
In my benchmarks I have used {\it libstdc++} implementation\cite{std::vector_impl} of std::vector. With default allocator, all allocation are done with {\it operator new} and deallocations with {\it operator delete}. Reallocation is done using scheme {\it allocate, move, deallocate}, where {\it allocate} and {\it  deallocate} phases are done by a given allocator. Growth factor is constant and is equal 2.

\section{folly::fbvector}
{\it Folly::fbvector}\cite{folly::fbvector_impl} is a part of {\it folly} library developed by Facebook. It has similar interface for allocator to {\it std::vector}, yet by default it utilizes {\it jemalloc} for allocations, deallocations and reallocations\cite{folly::fbvector_description}. It is worth noting that {\it folly::fbvector} utilizes {\it jemalloc xallocx} function, which tries to reallocate memory in place. Growth factor depends on size of current array. Initial growth is to at least 64 bytes, probably to fill whole cache line. For non in place reallocations (small ones) and big memory blocks (at least 4096 * 32 bytes) growth factor is equal 2. Otherwise growth factor is equal 1.5, as it allows to reuse previously allocated memory. {\it Folly::fbvector} developers believe that such strategy is more memory friendly and efficient\cite{folly::fbvector_description}. Additionally {\it folly::fbvector} uses {\it memcpy} to move objects with type decorated by {\it folly::IsRelocatable}.

\section{boost::container::vector}
{\it Boost::container::vector}\cite{boost::container::vector_impl} is the least specialized version of vector in {boost::container}. It has lower exception guarantees in order to improve performance of container\cite{boost_exceptions}. As a default allocator it uses boost version of dlmalloc\cite{dlmalloc}, which allows {\it boost::container::vector} to expand memory block forward as well as backward. To achieve this, allocator stores chain of allocations instead of single allocation block. Growth factor by default is equal 1.5, but it is possible to change its value at compilation time. In my opinion {\it boost::container::vector} has the most complex implementation out of those presented in this chapter.

\section{eastl::vector}
{\it Eastl::vector}\cite{eastl::vector_impl} is part of Electronic Arts Standard Template Library (EASTL) developed by Electronic Arts company. EASTL was designed especially as a game development library. It is considered to be more suited for console platforms\cite{eastl_faq} than other STL implementations. Default allocator in EASTL requires from user to define global {\it eastl} version of {\it operator new} that would be used for allocations. In my benchmarks I defined it to use standard version of {\it operator new}. {\it Eastl::vector} implementation is simple and similar to {\it std::vector} one, it also utilizes {\it allocate, move, deallocate} scheme. Yet it contains only EASTL version of STL functions. Growth factor is constant and is equal 2.  


\chapter{Rvector implementation}
{\it Rvector}\cite{rvector_impl} implements {\it std::vector} C++17 interface with few minor differences. Most important one is that {\it rvector} does not have any exception guarantees. Yet, as {\it gcc} (and {\it Clang}) follows Itanium ABI\cite{Itanium_ABI} in regard of exception handling, other vector implementations have a guarantee of zero overhead when exception throwing does not occur. \\
Main feature of {\it rvector} is use of syscall {\it mremap} to do reallocations. To make it possible all allocation of size greater than {\it page size} (which is 4KB) are done using syscall {\it mmap}. Smaller ones are done using {\it malloc} to reduce space consumption overhead, in that case standard {\it allocate, move, deallocate} scheme is utilized.

\begin{lstlisting}[caption=rvector allocation]
template<typename T>
T* allocate(size_type n) {
	if(n > map_threshold<T>)
    	return (T*) mmap(NULL, n*sizeof(T), 
                PROT_READ | PROT_WRITE,
                MAP_PRIVATE | MAP_ANONYMOUS,
                -1, 0);
    else
    	return (T*) malloc(n*sizeof(T));
}
\end{lstlisting}
In {\it rvector} array handling depends on element type traits. Especially important is whether certain type is trivially movable or not. To achieve compilation time dispatching I have used SFINAE feature with such policies:

\begin{lstlisting}[caption=SFINAE policies]
template<typename T, typename R = void>
using T_Move = std::enable_if_t<
				std::is_trivially_move_constructible<T>::value, R>;
template<typename T, typename R = void>
using NT_Move = std::enable_if_t<
				!std::is_trivially_move_constructible<T>::value, R>;
\end{lstlisting}
{\it Rvector} uses {\it mremap} in two different ways. For trivially movable types, MREMAP\_MAYMOVE flag is passed, which allows {\it mremap} to reallocate memory block to other address. Because it {\it mremap} does reallocation by changing page table mapping from virtual address to memory page\cite{mremap}, it is done in constant time, and thanks to MREMAP\_MAYMOVE flag this reallocation is always successful. 

\begin{lstlisting}[caption=rvector trivial type reallocation]
template<typename T>
T_Move<T, T*> realloc_(T* data, 
						size_type length, 
						size_type capacity, 
						size_type n) {
	// move between malloc and mmap allocations
	if((n > map_threshold<T>) != (capacity > map_threshold<T>)) {
        T* new_data = allocate<T>(n);
        memcpy(new_data, data, length * sizeof(T));
        deallocate(data, capacity);
        return new_data;
    }
    else {
        if(capacity > map_threshold<T>)
        	return (T*) mremap(data, capacity*sizeof(T), 
                    		n*sizeof(T), MREMAP_MAYMOVE);
        else
        	return (T*) realloc(data, n*sizeof(T));
    }
}
\end{lstlisting}
For non-trivially movable types, MREMAP\_MAYMOVE cannot be used. Without this flag {\it mremap} tries to expand memory block in place. This operation may fail, when there is not enough space in front of provided address. In that case standard reallocation is done.

\begin{lstlisting}[caption=rvector nontrivial type reallocation]
template<typename T>
NT_Move<T, T*> realloc_(T* data, 
						size_type length, 
						size_type capacity, 
						size_type n) {
    if(capacity > map_threshold<T>) { // try mremap fast reallocation
        void* new_data = mremap(data, capacity*sizeof(T), 
                    		n*sizeof(T), 0);
        if(new_data != (void*)-1)
        	return (T*) new_data;
    }
    T* new_data = allocate<T>(n);
    std::uninitialized_move_n(data, length, new_data);
    destruct(data, data + length);
    deallocate(data, capacity);
    return new_data;
}
\end{lstlisting}
Growth factor factor is equal to 2. Allocation size is also fixed by such function: 

\begin{lstlisting}[caption=fix capacity]
template <typename T>
size_type fix_capacity(size_type n) {
	// minimal allocation is 64 bytes as in folly::fbvector
	if(n < map_threshold<T>)
        return std::max(64/sizeof(T), n);
    // if requested capacity is greater than page size,
    // it is rounded to the next multiple of page size
    return map_threshold<T> * (n/map_threshold<T> + 1);
}
\end{lstlisting}
Each public function have been unit tested, using {\it gtest} library\cite{rvector_tests}. Tests are run with few different object types, with one of them being custom {\it TestType} designed to check correctness of object creation and destruction in {\it rvector}.

\begin{lstlisting}[caption=TestType, multicols=2, label=TestType_impl]
struct TestType
{
int n;
int* p;
static int aliveObjects;

TestType(int a = 5, 
		 int b = 1) noexcept
: n(a),
p(new int(b)) {
	aliveObjects++;
}

TestType(const TestType& other) noexcept
: n(other.n),
p(new int(*other.p)) {
	aliveObjects++;
}

TestType(TestType&& other) noexcept
: n(other.n),
p(other.p) {
	aliveObjects++;
	other.p = nullptr;
}

~TestType() {
	delete p;
	aliveObjects--;
}

TestType& operator = 
	(const TestType& other) noexcept {
	n = other.n;
	if(!p) p = new int();
	*p = *other.p;
	return *this;
}

TestType& operator = 
	(TestType&& other) noexcept {
	n = other.n;
	std::swap(p, other.p);
	return *this;
}
...
};
\end{lstlisting}
Tests consider trivially and non-trivially movable types, and small ({\it malloc} allocations) and big ({\it mmap} allocations) sizes, to check all branches of {\it rvector} memory handling.

\chapter{Benchmarks environment}
\section{System environment}
All tests have been performed on Intel(R) Core(TM) i7-6500U CPU @ 2.50GHz processor, with 8GB of RAM. Operating system: Ubuntu 16.04.5 LTS, kernel version Linux 4.4.0-141-generic. Compiler used was g++ 7.4.0, with optimization level: -O3.
\section{VectorEnv}
In order to benchmark vectors in complex, more realistic situations I have designed vector benchmark environment template.

\begin{lstlisting}[caption=VectorEnv declaration]
template <template<typename> typename V, typename... Ts>
class VectorEnv;
\end{lstlisting}
For given container type (e.g. {\it std::vector, rvector}) and element types it creates 
\lstinline{std::tuple<V<V<Ts>>...> env}
will contain all vectors created during benchmark. 
With created environment simulation can be run for given number of iterations.
At each iteration, for each element type, random action is dispatched.
\begin{lstlisting}[caption=RunSimulation]
double RunSimulation(int iter = 1000, std::string name) {
	BenchTimer bt(name);
	while(iter--) {
		(dispatch_action<Ts>(), ...);
	}
	return bt.check();
}
\end{lstlisting}
To gather time data I used custom class {\it BenchTimer}, where objects of that type utilize RAI pattern to save time spent in their scope;

\clearpage
\section{Simulation actions}
\begin{itemize}
\item [construct action]
\begin{lstlisting}[caption=construct action]
template <typename T>
void construct_action() {
	auto& typed_env = std::get<V<V<T>>>(env);
	std::uniform_int_distribution<> q_dist(1, 3);
	std::uniform_int_distribution<> size_dist(1, 1000);
	int q = q_dist(gen);
	
	while(q--) {
		int size = size_dist(gen);
		typed_env.emplace_back();
		BenchTimer bt("construct");
		while(size--)
			typed_env.back().emplace_back();
	}
}
\end{lstlisting}
Construct action creates few small vectors. Elements are {\it emplaced} one by one in order to check speed of small memory block reallocations. 

\item [push\_back action]
\begin{lstlisting}[caption=push\_back action]
template <typename T>
void push_back_action() {
	auto& typed_env = std::get<V<V<T>>>(env);
	if(typed_env.size() == 0) {
		construct_action<T>();
		return;
	}
	std::uniform_int_distribution<> 
				q_dist(1, typed_env.size() / 3 + 1);
	std::uniform_int_distribution<> 
				pick_dist(0, typed_env.size()-1);
	std::uniform_int_distribution<> 
				size_dist(1, 100000);
	int q = q_dist(gen);
	
	BenchTimer bt("push_back");
	while(q--) {
		int pick = pick_dist(gen);
		int size = size_dist(gen);
		while(size--)
			typed_env[pick].emplace_back();
	}
}
\end{lstlisting}
Adds many elements (up to 100000) to part of environment (up to 1/3 with replacement). Other actions randomize the process in similar manner.

\item [pop\_back action]
Pops elements (up to size of picked vector) of part of environment (up to 1/3 with replacement). 

\item [copy action]
Copies up to 3 vectors (with replacement) of environment.

\item[insert action]
Inserts elements (into random position of picked vector, up to 10000) into part of environment (up to 1/3 with replacement).

\item[erase action]
Erases elements (between random positions in picked vector) from part of environment (up to 1/3 with replacement).
\end{itemize}

\section {Experiments}
\begin{lstlisting}[caption=experiment function]
template <template<typename> typename V, typename... Ts>
void experiment(std::string name, 
				int max_it = 1000, 
				int tests = 10) {
	for(int iter = 100; iter <= max_it; iter += 100) {
		std::vector<double> test_res;
		for(int seed = 12345512; seed < 12345512 + tests; seed++) 
		{
			VectorEnv<V, Ts...> v_env(seed);
			test_res.push_back(v_env.RunSimulation(iter));
		}
		// data gathering and saving
		...
}
\end{lstlisting}
For each tested vector implementation I have run experiment function with few different element types (Ts...). 

\begin{figure}[h!]
\includegraphics{plots/capacity_<std::string,int,std::array<int,10>>}
\caption{Needed space for element types \lstinline{int, std::string, std::array<int, 10>} [bytes].}
\label{space_consumption}
\end{figure}

\begin{figure}[h!]
\includegraphics{plots/length_<std::string,int,std::array<int,10>>}
\caption{Maximum vector length for element types \lstinline{int, std::string, std::array<int, 10>} [bytes].}
\label{vector_length}
\end{figure}


As it can be seen on Figure \ref{space_consumption} average number of bytes needed for environment (sum of lengths of vectors) is increasing with iterations. Maximum length of single vector stabilizes after few hundred iterations \ref{vector_length}. Benchmarks with more iterations will show vector operations efficiency on more fragmented memory, as the total number of vectors is increasing.

\chapter{Benchmarks results}

I have benchmarked {\it std::vector, rvector, folly::fbvector, boost::container::vector, eastl::vector} using my {\it VectorEnv} framework. All vectors had growth factor equal 2, except {\it folly::fbvector} which has custom growth factor. I have done experiments for each of those with such element types:
\begin{itemize}
\item \lstinline{int}
\\
Size of \lstinline{int} is 4 bytes and it is POD.
\item \lstinline{std::array <int, 10>}
\\
Size of \lstinline{std::array <int, 10>} is 40 bytes and it is POD.
\item \lstinline{TestType}
\\
Size of \lstinline{TestType} is 16 bytes, it non-trivial. It is worth noting that each constructor and assignment operator has {\it noexcept} attribute, which allows vector implementations to use moving constructor instead of copy in case of reallocation. Also moving constructor is much faster than copy as it does not allocate memory.
\item \lstinline{std::string, int, std::array<int, 10>}
\\
During this benchmark, all element types will be tested at the same time. Size of \lstinline{std::string} is 32 and it is not trivial type. As in \lstinline{TestType} its moving constructor has {\it noexcept} attribute.
\end{itemize}

\section{Element types: int}
Integer is one of tested element types to check how efficient vector is for trivial, small objects. Benchmarks have shown that {\it std::vector} with simple implementation containing intrinsic operations did much better than more complex ones. 

\begin{figure}[h!]
\includegraphics{plots/int_pop_back}
\caption{Pop\_back action time for element type \lstinline{int}.}
\label{int_pop_back}
\end{figure}

It turns out that {\it folly::fbvector} and {\it boost::container::vector} implementation of pop\_back checks whether object is trivially destructible at runtime instead of at compilation time. Results of pop\_back action benchmark \ref{int_pop_back} show that branching at pop\_back makes it much slower.

\begin{figure}[h!]
\includegraphics{plots/int_copy}
\caption{Copy action time for element type \lstinline{int}.}
\label{int_copy}
\end{figure}

\begin{figure}[h!]
\includegraphics{plots/int_copy_constrained}
\caption{Copy action time for element type \lstinline{int} with size constrained to 100.}
\label{int_copy_constrained}
\end{figure}

{\it Rvector} falls behind other implementation at copy action benchmark as it can be seen at Figure \ref{int_copy}. As for \lstinline{int} element type (and all other trivially copyable types), copy constructor is equivalent to single allocation and {\it memcpy}. During benchmarks {\it rvector} mainly uses {\it mmap} to allocate memory block for copy, which is much slower that allocation with other memory managers. Figure \ref{int_copy_constrained} shows that {\it rvector} is as fast as other vectors when it does not use {\it mmap} to allocate memory for copy, because size of copy is constrained to 400 bytes.

\begin{figure}[h!]
\includegraphics{plots/int_push_back}
\caption{Push\_back operation time for element type \lstinline{int}.}
\label{int_push_back}
\end{figure}

{\it Rvector} did better that other vectors at push\_back action benchmark, yet difference is not as big as for other element types. Because vectors did not operate on big memory blocks, even at large number of elements, reallocation optimizations did not induce much speed up.

\begin{figure}[h!]
\includegraphics{plots/int_Simulation}
\caption{Total simulation time for element type \lstinline{int}.}
\label{int_simulation}
\end{figure}

Because all vector implementation got similar results on erase and insert action results, overall score is mainly influenced by push\_back and pop\_back action.

\section{Element type: std::array}
In order to benchmark vectors on big and trivial objects, I have used as \lstinline{std::array<int, 10>} element type. Main difference between \lstinline{std::array<int, 10>} and \lstinline{int} benchmarks, are those on push\_back action. As it can be seen on Figure \ref{array_push_back}, {\it rvector} did all push\_back operations in 42 seconds, yet runner-up {\it eastl::vector} did them in 56 seconds. Reallocation using {\it mremap} are much faster for big memory blocks. Figure \ref{array_simulation} shows total simulation time for \lstinline{std::array<int, 10>} element type.

\begin{figure}[h!]
\includegraphics{plots/std::array<int,10>_push_back}
\caption{Push\_back operation time for element type \lstinline{std::array<int, 10>}.}
\label{array_push_back}
\end{figure}

\begin{figure}[h!]
\includegraphics{plots/std::array<int,10>_insert}
\caption{Insert operation time for element type \lstinline{std::array<int, 10>}.}
\label{array_insert}
\end{figure}

\begin{figure}[h!]
\includegraphics{plots/std::array<int,10>_Simulation}
\caption{Total simulation time for element type \lstinline{std::array<int, 10>}.}
\label{array_simulation}
\end{figure}

\clearpage
\section{Element type: TestType}
TestType class have been defined in Listing \ref{TestType_impl}. It is not trivial and its move constructor is significantly faster than copy constructor, as it does not allocate memory. In this benchmark  {\it rvector} will try to expand memory in place using {\it mremap} without MREMAP\_MAYMOVE flag. On the one hand, fast reallocation will not occur always, as it did with trivial types. On the other hand, normal reallocations are more expensive, because objects must be moved with their move or copy constructors, not with simple {\it memcpy}.

\begin{figure}[h!]
\includegraphics{plots/TestType_Simulation}
\caption{Total simulation time for element type \lstinline{TestType}.}
\label{TestType_push_back}
\end{figure}

It turns out that gain from {\it mremap} is not as significant as for \lstinline{std::array<int, 10>}, yet it is getting greater, when larger memory blocks are reallocated. Results for insert action can be seen on Figure \ref{TestType_insert}. 
\\


As for construct action, {\it rvector} got much worse results that all other vectors. Using a profiler I found out that {\it malloc} functions called by {\it rvector} needed around 60\% more CPU cycles to finish than for {\it std::vector} calls. On the other hand {\it rvector} did significantly better on copy action. I assumed that such behavior must be connected with use of {\it mmap}. To test this hypothesis I benchmarked {\it rvector} with {\it mmap/mremap} optimization turned off. Results of construct and copy action on such setup can be seen of Figures \ref{TestType_construct_malloc} and \ref{TestType_copy_malloc} respectively.

\begin{figure}[h!]
\includegraphics{plots/TestType_push_back}
\caption{Push\_back action time for element type \lstinline{TestType}.}
\label{TestType_push_back}
\end{figure}

\begin{figure}[h!]
\includegraphics{plots/TestType_insert}
\caption{Insert action time for element type \lstinline{TestType}.}
\label{TestType_insert}
\end{figure}

\begin{figure}[h!]
\includegraphics{plots/TestType_construct}
\caption{Construct action time for element type \lstinline{TestType}.}
\label{TestType_construct}
\end{figure}

\begin{figure}[h!]
\includegraphics{plots/TestType_copy}
\caption{Copy action time for element type \lstinline{TestType}.}
\label{TestType_copy}
\end{figure}


\begin{figure}[h!]
\includegraphics{plots/TestType_construct_malloc}
\caption{Construct action time for element type \lstinline{TestType}. {\it Rvector} without {\it mmap}.}
\label{TestType_construct_malloc}
\end{figure}

\begin{figure}[h!]
\includegraphics{plots/TestType_copy_malloc}
\caption{Copy action time for element type \lstinline{TestType}. {\it Rvector} without {\it mmap}.}
\label{TestType_copy_malloc}
\end{figure}

\section{Element types: std::string, int, std::array}
In this benchmark 3 different element types has been tested at the same moment. It makes this benchmark the most computational heavy and memory consuming of all benchmarks presented in this paper. Single simulation in a peak moment had a working space of around 14GB. As it is more than available physical memory,  simulation show results of vector during {\it throttling}.

\begin{figure}[h!]
\includegraphics{plots/general_Simulation}
\caption{Total simulation time for element types \lstinline{std::string, int, std::array}.}
\label{general_simulation}
\end{figure}

As it can be seen on Figure \ref{general_simulation}, {\it rvector} gets significant advantage over other vector implementations. Every action except construct is done faster with {\it rvector}, which can be seen on the plots below. It is worth noting that even such actions as erase and pop\_back, that do not induce reallocation, took less time to execute with use of {\it rvector}. 

\begin{figure}[h!]
\includegraphics{plots/general_push_back}
\caption{Push\_back action time for element types \lstinline{std::string, int, std::array}.}
\label{general_push_back}
\end{figure}

\begin{figure}[h!]
\includegraphics{plots/general_copy}
\caption{Copy action time for element types \lstinline{std::string, int, std::array}.}
\label{general_copy}
\end{figure}

\begin{figure}[h!]
\includegraphics{plots/general_erase}
\caption{Erase action time for element types \lstinline{std::string, int, std::array}.}
\label{general_erase}
\end{figure}

\begin{figure}[h!]
\includegraphics{plots/general_construct}
\caption{Construct action time for element types \lstinline{std::string, int, std::array}.}
\label{general_construct}
\end{figure}

\chapter{Conclusions}
Considering results from benchmarks on small and large POD types, one can conclude that gain from using {\it mremap} as reallocation is more significant with greater memory block to be reallocated. It also the case for non-POD types, yet advantage is drastically lower. Surprisingly {\it rvector} turned out to be much more effective when memory resources are running low. It is also worth noting that even though {\it rvector} has much simpler implementation than other considered vectors, it did not fall greatly behind on benchmark when it's main optimizations were turned off.
\\
As a summary, here are features and advantages of {\it rvector}:
\begin{itemize}
\item Much faster reallocation when element type is POD, which is helpful for heavy use of push\_back or insert.
\item For non-POD types {\it mremap} in place still speeds up reallocation, yet not as much as it does for POD type.
\item {\it Rvector} outperforms other vector implementations during throttling
\item Very simple implementation, which yet requires C++17.
\item No dependencies, but it is implemented only for Linux based systems as it explicitly uses {\it mmap, mremap} and{\it munmap} syscalls.
\item Additional methods:
	\begin{itemize}
	\item fast\_push\_back, fast\_emplace\_back - does not check capacity bound. Can be used in {\it reserve, fill} pattern.
	\item safe\_pop\_back - pop\_back which check if target vector is empty, if so, does nothing.
	\end{itemize}
\item Compliant with C++17 {\it std::vector} interface, including deduction guides for constructors.
\end{itemize}

%%%%% BIBLIOGRAFIA
\begin{thebibliography}{1}
\bibitem{mremap} mremap syscall manual \url{http://man7.org/linux/man-pages/man2/mremap.2.html}
\bibitem{std::vector_impl} std::vector libstdc++ implementation \url{https://github.com/gcc-mirror/gcc/blob/master/libstdc%2B%2B-v3/include/bits/stl_vector.h}

\bibitem{folly::fbvector_impl} folly::fbvector implementation \url{https://github.com/facebook/folly/blob/master/folly/FBVector.h}
\bibitem{folly::fbvector_description} folly::fbvector description \url{https://github.com/facebook/folly/blob/master/folly/docs/FBVector.md}

\bibitem{boost::container::vector_impl} boost::container::vector documentation \url{https://www.boost.org/doc/libs/1_68_0/doc/html/boost/container/vector.html}
\bibitem{boost_exceptions} \url{https://www.boost.org/doc/libs/1_68_0/doc/html/container/cpp_conformance.html#container.cpp_conformance.vector_exception_guarantees}
\bibitem{dlmalloc} dlmalloc description \url{ftp://g.oswego.edu/pub/misc/malloc.c}

\bibitem{eastl::vector_impl} eastl::vector implementation \url{https://github.com/electronicarts/EASTL/blob/master/include/EASTL/vector.h}
\bibitem{eastl_faq} EASTL FAQ \url{https://rawgit.com/electronicarts/EASTL/master/doc/EASTL%20FAQ.html}

\bibitem{rvector_impl} rvector implementation \url{https://github.com/Bixkog/rvector/blob/master/rvector.h}
\bibitem{Itanium_ABI} Itanium ABI for exception handling \url{https://itanium-cxx-abi.github.io/cxx-abi/abi-eh.html}

\bibitem{rvector_tests} rvector tests \url{https://github.com/Bixkog/rvector/blob/master/test.cpp}
\end{thebibliography}


\listoffigures
\lstlistoflistings
\end{document}
